\chapter{Introduction}
\label{cha:introduction}
Recent astonishing advances on sensing modalities, data mining, communication and computation design methodologies enable us to encode multi-system interactions in both physical and cyber domains by the construction of cyber physical systems (CPS). CPS principles play the primary role in building up the foundations for a variety of emerging domains such as precise and personalized medicine, intelligent and autonomous vehicles and internet-of-things (IoT). However, enabling a closed-loop control over the CPS systems requires a systematic understanding of the synergistic coupling between physical and cyber processes. This poses major challenges in construction of the theoretical and computational foundations for making CPS a reachable reality.
In the computational domain, advanced scientific and engineering investigations in CPS systems generate exorbitantly large amount of data that is much beyond our accessible computing power. For instance, the data generated by genomics applications in precise medicine are projected to exceed any other Big Data applications with scale of zeta-byte by year of 2025. Processing astronomical volume of data requires computing capabilities ranging from 2-1000 trillion CPU hours due to the excessive resource locking, data contention and poorly exploited fine-grain parallelism in well-established computing paradigms (e.g., CPU, GPGPU). The fact that CPS applications are rich in non-stationary behaviors with lack of regular communication structures (e.g., physiological and biological system learning and control) exacerbates the drawbacks of parallelization techniques, memory wall, or the need for multiple instruction multi-data (MIMD) execution. Overcoming these challenges requires revolutionary design methodologies that break the hardware-software boundary to provide algorithmic breakthroughs for data movement, energy efficiency, memory wall, and extreme concurrency challenges. 

In the theoretical domain, the mathematical understanding, description, prediction and control of physical processes poses critical challenges in design and optimization of CPS systems that are highly reliable and robust with bounded performance margins. The irregular and non-stationary dynamics of CPS applications contradicts the traditional statistical memoryless assumption (e.g., Markovian processes, short-term memory) and cannot be captured by integer-order calculus. Instead, they exhibit long-term memory and can only be accurately modeled by fractal-order differential equations. For instance, inferring causal and higher-order complex relations in physiological and biological systems (e.g., cardiovascular, muscular or neural systems) from unstructured time varying data requires a spatio-temporal fractal model that considers multivariate dependency and a power-law temporal correlations. The failure to capture these properties of CPS applications in a robust and real-time fashion not only prevents its practical employment, but can also lead to irreversible consequences especially in presence of noisy and limited measurements, system anomalies, environmental uncertainties and malicious attacks. 

To tackle these challenges, my PhD research focuses on the following primary research tasks:
(1). How can we leverage a new design methodology to be capable of processing and mining the exscale data with optimal computation and communication architectures for CPS applications of tremendous heterogeneity? 
(2). What are the compact yet accurate construction of mathematical models that are able to capture the non-Markovian, non-stationary and non-linear system dynamics? 
To address the first task, we propose radical design methodologies for the exascale computation and communication architectures, making the following contributions: (1) We develop a LLVM-compiler framework based architecture-independent application profiling engine to learn the CPS application behaviors and inter-dependent structure in real-time. (2) We propose a novel model of computation (MoC) that describes the identified CPS applications behaviors and structures as a directed dynamical weighted graph (DDWG) in terms of required computation capabilities, data movement, storage requirements and timing characteristics to support the discovery of the optimal fine-grained parallelism and enables an automatic top-down synthesis of best-fit architectures with 10x performance improvement given the same energy budget. (2) To tackle the dynamic nature of CPS applications (e.g., adaptive multi-scaling, coupled multi-scale multi-physics), we develop traffic-aware routing algorithm with the support for node-cooperation network-coding communication and sub-modular algorithm with bounded optimality for NoC reconfiguration to achieve 100-fold throughput improvement and enable energy-efficient terabyte (TB) data movement. (3) We propose a general hardware accelerator synthesis framework by combining (1) and (2) to facilitate automated application-to-architecture top-down design methodology for CPS systems.

 To address the second task, we embrace the novel mathematical framework in modeling and control theory. More specifically: (1) We pioneer a dynamical systems approach to the problem of tracking both muscular and neural activity, and demonstrated that the current modeling approaches (e.g., ARMA. ARFIMA) of muscular and neural activities either ignored the long-term dependency in system dynamics or spatial coupling among system components. (2) We propose to describe the evolution of biological systems by construction of discrete-time fractional-order systems and demonstrate much improved performance in terms of goodness-of-fit against modeling approaches with short-term memory assumptions. (3) We show that the problem of determining the minimum number of sensors such that the global dynamics described by the fractional-order differential equation can be recovered from the collected data (i.e., observability) is NP-hard. Subsequently, we propose a polynomial algorithm to obtain suboptimal solutions with optimality guarantees and are robust in presence of modeling errors, process and measurement noise. (4) We investigate the state-of-art wearables that monitors brain activities (e.g., Emotiv) and show that the current placement of EEG sensors on the scalp is not optimal with current setup.

\Blindtext[1]
