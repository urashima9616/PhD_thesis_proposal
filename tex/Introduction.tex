\chapter{Introduction}
\label{cha:introduction}
% ------Abstract of ICCPS

The physical world is deeply rooted in complex interactions among synergetically coupled components of systems at different scales. Recent astonishing advances on cognitive sensing modalities, data science, machine learning frameworks, communication and computation design methodologies not only enable us to capture, analyze and impact such multi-system interactions with remarkably improved accessibility, operability, resolution and sophisticated mathematical and processing frameworks, but also calls for a radically novel integration of state-of-art techniques in both \textit{theoretical} and \textit{computational} domains to endow the cyber-physical systems (CPS) with built-in \textit{intelligence} and realtime processing capabilities. On one hand, the envisioned intelligent CPS is prescribed by the urgent demands for a revolutionary shift from the manually intervened analyzing paradigms to a self-aware, self-optimized and self-configured systems of extreme autonomy to facilitate adaptive sensing (i.e., data collection), learning (i.e., data processing) and decision making (i.e., control) processes with minimal human supervision over extended periods of operation. On the other hand, porting the embedded intelligence to devices with strict real-time processing constraints raises concerns with respect to the applicability of the well-established general-purpose computing paradigms (e.g., CPU/GPU array or clusters, data centers, cloud-computing) to application scenarios that have i) exorbitantly large amount of raw data, ii) limited  or insufficient communication bandwidth and/or iii) strict power,  thermal budget and timing constraints. These objectives can not be reached without an integrated mathematical and computational framework that is able to  i) capture and understand the complex interdependent structure of CPS processes rich in stochasticity and ii) provides real-time processing capabilities that can be encompassed by on-board computing powers dynamically optimized for interested CPS applications. Towards this end, several key research challenges in both theoretical and computational domains have to be sufficiently addressed by this PhD work.

From the theoretical perspective, a systematic understanding of inter-coupling (e.g., causal influence,  coupling directionality and structure, spatial and temporal distribution) among heterogeneous entities involved in the evolution of the investigated physical process is a prerequisite. From microbial communities, human physiological and biological systems to large scale intelligent networked systems (e.g., smart grid, metropolitan traffic management systems) and social networks, complex interdependent systems usually display multi-scale \textit{spatio-temporal} dynamics that are frequently classified as \textit{non-linear, non-Gaussian, non-ergodic}, and/or \textit{fractal}~\cite{bogdan2015cyber,bogdan2015cyber_2,ghorbani2013cyber,xue2016spatio,kumar2013mobile}. Distinguishing between the sources of nonlinearity, identifying the nature of fractality (space versus time) and encapsulating the non-Gaussian characteristics into dynamic causal models remains a major challenge for studying complex systems. More specifically, a number of outstanding problems for constructing mathematical models of complex systems should be considered: (\textit{i}) How can we distinguish between spatial and temporal nonlinearity and how can we construct mathematical models (i.e., dynamical equations) that capture the spatio-temporal statistical characteristics of complex systems? (\textit{ii}) How can we identify the mathematical expressions for the nonlinear models of complex systems and determine the degree of nonlinearity that should be accounted for without incorporating unnecessarily many nonlinear terms? (\textit{iii}) How can the power law and non-Gaussian properties of the magnitudes observed in many time series impact the degree of nonlinearity? (\textit{iii}) How one can interpret the asymmetry observed in the time series realization of various processes and estimate the amount of information gained from analyzing the spatio-temporal complexity present in time series? The failure to address these problem lead to a modeling of system evolution that either suffers from biased characterization of the dynamics (e.g., under-fitting) or hardly generalize to endow the learned model with predictive powers (e.g., over-fitting).

Moreover, the mathematical understanding, description, prediction and control of physical processes poses critical challenges in design and optimization of CPS systems that are highly reliable and robust with bounded performance margins. The irregular and non-stationary dynamics of CPS applications contradicts the traditional statistical memoryless assumption (e.g., Markovian processes, short-term memory) and cannot be captured by integer-order calculus. Instead, they exhibit long-term memory and can only be accurately modeled by fractal-order differential equations. For instance, inferring causal and higher-order complex relations in physiological and biological systems (e.g., cardiovascular, muscular or neural systems) from unstructured time varying data requires a spatio-temporal fractal model that considers multivariate dependency and a power-law temporal correlations. The failure to capture these properties of CPS applications in a robust and real-time fashion not only prevents its practical employment, but can also lead to irreversible consequences especially in presence of noisy and limited measurements, system anomalies, environmental uncertainties and malicious attacks. 

All these issues are further exacerbated by the fact that the current deployment of intelligent CPS highly rely on large scale general-purpose computing platforms (data-center scale clustered CPUs, arrays of GPUs or GPGPUs). These traditional computing power are exorbitantly power consuming and require extensive maintenance, adjusting and monitoring efforts. The flexibility of these well-established platforms endows the fast-forwarding CPS applications with minimized development efforts and highly dynamical supporting communities. However, this also prevents its portability to power-limited devices and instruments with strict constraints (e.g., edge devices). Furthermore, advanced scientific and engineering investigations in CPS systems generate exorbitantly large amount of data that is much beyond easily accessible computing power. Relaying large volume of unstructured and under-explored raw data back to a processing fusion not only drains unnecessarily huge amount of energies and computing powers, but also place extremely high demands on the bandwidth of communication bandwidth especially when a prompt decision and feedback is required. For instance, the data generated by genomics applications in precise medicine are projected to exceed any other Big Data applications with scale of zeta-byte by year of 2025. Processing astronomical volume of data requires computing capabilities ranging from 2-1000 trillion CPU hours due to the excessive resource locking, data contention and poorly exploited fine-grain parallelism in well-established computing paradigms (e.g., CPU, GPGPU). The fact that CPS applications are rich in non-stationary behaviors with lack of regular communication structures (e.g., physiological and biological system learning and control) exacerbates the drawbacks of parallelization techniques, memory wall, or the need for multiple instruction multi-data (MIMD) execution. All these challenges sum up to the urgent need for a shift to design methodologies that optimize for target applications (rather than applications in general) to break the hardware-software boundary to provide remarkable improvement on data movement, energy efficiency, memory wall and concurrency under rich uncertainties with enhanced robustness. 

To embrace the complexity in CPS design and tackle these challenges, this PhD work propose to devote research efforts on the following two primary research topics:\\
\noindent \textbf{(Task 1). Exploration for compact yet accurate construction of mathematical models that are able to capture the non-Markovian, non-stationary and non-linear system dynamics in a wide specturm of CPS applications}\\
\textbf{(Task 2). A new top-down design methodology for CPS applications capable of processing and mining the exascale data with optimal computation and communication architectures.}\\
To address the theoretical challenges posed by Task 1 and 2, we embrace the novel mathematical framework in modeling and control theory and advocate for a top-down design methodology for computing and communication platforms. More specifically: \\
(1) In Chapter 2, we propose a new mathematical strategy for constructing compact yet accurate models of complex systems dynamics that aim to scrutinize the causal effects and influences by analyzing the statistics of the magnitude increments and the inter-event times of stochastic processes. We derive a framework that enables to incorporate knowledge about the causal dynamics of the magnitude increments and the inter-event times of stochastic processes into a multi-fractional order nonlinear partial differential equation for the probability to find the system in a specific state at one time. Rather than following the current trends in nonlinear system modeling which postulate specific mathematical expressions, this mathematical framework enables us to connect the microscopic dependencies between the magnitude increments and the inter-event times of one stochastic process to other processes and justify the degree of nonlinearity. In addition, the newly presented formalism allows to investigate appropriateness of using multi-fractional order dynamical models for various complex system which was overlooked in the literature.\\
(2) To validate the framework discussed in Chapter 2, we pioneer a dynamical systems approach to the problem of tracking both muscular and neural activity, and demonstrated that the current modeling approaches (e.g., ARMA. ARFIMA) of muscular and neural activities either ignored the long-term dependency in system dynamics or spatial coupling among system components in Chapter 3. We propose to describe the evolution of biological systems by construction of discrete-time fractional-order systems and demonstrate much improved performance in terms of goodness-of-fit against modeling approaches with short-term memory assumption.\\ 
(3) In Chapter 4, to allow optimized sensing strategies with minimized sensory data collection efforts in processes described by the proposed fractional-order dynamical system model, we study problem of determining the minimum number of sensors such that the global dynamics described by the fractional-order differential equation can be recovered from the collected data (i.e., observability). We also show the problem is NP-hard. We propose a polynomial algorithm to obtain suboptimal solutions with optimality guarantees and are robust in presence of modeling errors, process and measurement noise. By taking advantage of the proposed approach, we investigate the state-of-art wearables that monitors brain activities (e.g., Emotiv) and show that the current placement of EEG sensors on the scalp is not optimal with current setup.\\
(4) In Chapter 5, we propose a model-of-computation (MoC) based application profiling framework to understand the computation and communication requirement of the applications, which enables a novel design methodology for the exascale computation and communication architectures optimized for CPS in runtime.  More specifically, we make the following contributions: (i) We develop a LLVM-compiler framework based architecture-independent application profiling engine to learn the CPS application behaviors and inter-dependent structure in real-time. (ii) We propose a novel model of computation (MoC) that describes the identified CPS applications behaviors and structures as a directed dynamical weighted graph (DDWG) in terms of required computation capabilities, data movement, storage requirements and timing characteristics to support the discovery of the optimal fine-grained parallelism and enables a top-down synthesis of best-fit architectures with remarkable performance improvement given the same energy budget.\\
(5) In Chapter 6, we tackle the dynamic nature of CPS application traffics after learning the task structure and data movement properties in Chapter 5 to facilitate a runtime optimization for best-fit communication architecture that sustains the data processing (computation). Power-efficient data movement strategies for CPS applications are required to shift more power from communication toward computation. Towards this end, we first propose a novel hierarchical NoC architecture that exploits user-cooperated network coding (NC) concepts for improving system throughput especially under heavy collective traffics (e.g., multicast, broadcast). To improve this technique with runtime optimization capability that allows for synthesis of best-fit communication architecture in general cases, we propose a general reconfigurable Network-on-Chip (NoC) synthesis framework with guaranteed optimality given the target CPS application profiled by DDWG. This framework combined with the profiling technique discussed in Chapter 5 lays the foundation for the automated tow-down computing and communication optimization engine that can serve as the cognitive core for future intelligent CPS systems.\\

%\Blindtext[1]
