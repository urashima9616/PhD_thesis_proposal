Recent astonishing advances on cognitive sensing modalities, data science, machine learning frameworks, communication and computation design methodologies not only enable us to capture, analyze and impact such multi-system interactions with remarkably improved accessibility, operability, resolution and sophisticated mathematical and processing frameworks, but also calls for a radically novel integration of state-of-art techniques in both \textit{theoretical} and \textit{computational} domains to endow the cyber-physical systems (CPS) with built-in \textit{intelligence} and realtime processing capabilities. The  intelligent CPS is prescribed by the urgent demands for a revolutionary shift from the manually intervened analyzing paradigms to a self-aware, self-optimized and self-configured systems of extreme autonomy to facilitate adaptive sensing (i.e., data collection), learning (i.e., data processing) and decision making (i.e., control) processes with minimal human supervision over extended periods of operation. However, porting the embedded intelligence to devices with strict real-time processing constraints raises concerns with respect to the applicability of the well-established general-purpose computing paradigms (e.g., CPU/GPU array or clusters, data centers, cloud-computing) to application scenarios that have i) exorbitantly large amount of raw data, ii) limited  or insufficient communication bandwidth and/or iii) strict power,  thermal budget and timing constraints. These objectives can not be reached without an integrated mathematical and computational framework that is able to  i) capture and understand the complex interdependent structure of CPS processes rich in stochasticity and ii) provides real-time processing capabilities that can be encompassed by on-board computing powers dynamically optimized for interested CPS applications. \\
\indent In this proposal, we aim to address these challenges by embracing the novel mathematical framework in modeling and control theory and advocating for a top-down design methodology for computing and communication platforms. We propose a new mathematical strategy for constructing compact yet accurate models of complex systems dynamics that aim to scrutinize the causal effects and influences by analyzing the statistics of the magnitude increments and the inter-event times of stochastic processes. We derive a framework that enables to incorporate knowledge about the causal dynamics of the magnitude increments and the inter-event times of stochastic processes into a multi-fractional order nonlinear partial differential equation for the probability to find the system in a specific state at one time. Rather than following the current trends in nonlinear system modeling which postulate specific mathematical expressions, this mathematical framework enables us to connect the microscopic dependencies between the magnitude increments and the inter-event times of one stochastic process to other processes and justify the degree of nonlinearity. In addition, the proposed formalism allows to investigate appropriateness of using multi-fractional order dynamical models for various complex system which was overlooked in the literature. We validate the proposed modeling framework by showing that the multi-fractal dynamical equations learned from muscular, neural and vascular systems are accurately consistent with the system dynamics and possess predictive power with minimized sensory efforts.

To encompass the built-in intelligence and realtime processing capabilities within efficient computing and communication powers, we propose to advocate for a shift from ad-hoc architectural optimization in a bottom-up fashion to an automated application-driven top-down optimization flow by learning the target applications and allowing runtime configuration of computing and communication architectures. More precisely, we propose a model-of-computation (MoC) based application profiling framework to understand the computation and communication requirement of the applications. MoC describes the identified CPS applications behaviors and structures as a directed dynamical weighted graph in terms of required computation capabilities, data movement, storage requirements and timing characteristics to support the discovery of the optimal fine-grained parallelism for design of processing elements. To tackle the dynamic and irregular nature of communication workloads in CPS applications to facilitate a runtime optimization for best-fit communication architecture that sustains the data processing with maximized efficiency, we first propose a novel hierarchical NoC architecture that exploits user-cooperated network coding (NC) concepts for improving system throughput especially under heavy collective traffics (e.g., multicast, broadcast). Then we further take advantage of the MoC-based profiling framework and propose a general mathematical framework for reconfigurable Network-on-Chip (NoC) synthesis with guaranteed optimality.

In summary, this PhD work explores for compact yet accurate construction of mathematical models that are able to capture the non-Markovian, non-stationary and non-linear system dynamics in a wide spectrum of CPS applications and advocate for a new top-down design methodology for CPS applications capable of processing and mining the exa-scale data with optimal computation and communication architectures. With the theoretical and computational contribution combined, this PhD work aims to serve as the basis for the intelligent core of future autonomous CPS with self-understanding, self-configuration and self-optimization capabilities.

