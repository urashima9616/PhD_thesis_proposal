\chapter{Introduction}
\label{cha:introduction}
% ------Abstract of ICCPS
From microbial communities, human physiology to social and bio- logical/neural networks, complex interdependent systems display multi-scale spatio-temporal pa erns that are frequently classi ed as non-linear, non-Gaussian, non-ergodic, and/or fractal. Distin- guishing between the sources of nonlinearity, identifying the na- ture of fractality (space versus time) and encapsulating the non- Gaussian characteristics into dynamic causal models remains a ma- jor challenge for studying complex systems. In this paper, we pro- pose a new mathematical strategy for constructing compact yet ac- curate models of complex systems dynamics that aim to scrutinize the causal e ects and in uences by analyzing the statistics of the magnitude increments and the inter-event times of stochastic pro- cesses. We derive a framework that enables to incorporate knowl- edge about the causal dynamics of the magnitude increments and the inter-event times of stochastic processes into a multi-fractional order nonlinear partial di erential equation for the probability to  nd the system in a speci c state at one time. Rather than follow- ing the current trends in nonlinear system modeling which pos- tulate speci c mathematical expressions, this mathematical frame- work enables us to connect the microscopic dependencies between the magnitude increments and the inter-event times of one stochas- tic process to other processes and justify the degree of nonlinearity. In addition, the newly presented formalism allows to investigate appropriateness of using multi-fractional order dynamical models for various complex system which was overlooked in the literature. We run extensive experiments on several sets of physiological pro- cesses and demonstrate that the derived mathematical models o er superior accuracy over state of the art techniques.
%--------IoT introduction
Recent astonishing advances on sensing modalities, energy-sensitive data processing, communication and storage system design methodologies lead to capabilities to encode the tremendously complicated interactions among the i) physical entities, ii) humanity-related activities and iii) man-made devices of enormous heterogeneity, by constructing a seamless continuum of networks.  We are now embracing  a physical world unprecedentedly tight-linked via an extremely extended web of smart devices that are rich in capabilities of sensing, computing, communicating and control. As envisioned, objects, ranging from tiny physical entities, e.g., cells and microbial clusters, to giant facilities, e.g., power plants and nation-wide power grids, will be uniquely addressable and identifiable~\cite{zanella2014internet}. Their interactions are captured (i.e. sensing), exchanged (i.e., communication) and analyzed (i.e., computation) by devices that are enormously heterogeneous yet synergistically collaborating through networked systems, hence forming an internet of "things" (IoT).

 Apparently, such vision has not yet become a close reality that is accessible on an everyday basis. However, IoT is already unveiling its transformative role in rebooting the fundamental ways of our interaction with the physical world~\cite{stankovic2014research,pajic2015attack,fawzi2014secure,clark2016submodularity}. For instance,  smart grids that consist of numerous networked power plants are already extensively deployed to dynamically capture the spatial-temporal variations in user demands and optimize the power plant management for maximization of the efficiency and robustness; wearable physiological sensors and actuators are actively adopted to monitor the crucial bio-markers and provide prompt medical response to alleviate or reverse potential health risks~\cite{bogdan2015cyber,bogdan2015cyber_2,ghorbani2013cyber,xue2016spatio,kumar2013mobile}; smart portable devices are able to run highly varied applications with full awareness of user contexts, which is enabled by vastly varied embedded ambience sensors (e.g., gyroscope, accelerometer, GPS, temperature and lighting sensors) as well as advanced data processing technologies (e.g., data mining, deep learning).\\
\indent Physical world is deeply rooted in seamless interactions among entities from vastly varied domains. In contrast to intrinsically connected physical world, the prior research efforts on IoT are still very fragmented and, very often, confined to disjoint explorations in different application, architectural, security, service, protocol and economical domains, which poses the following challenges in making IoT a reachable reality in near future: 

\noindent\textbf{i) How are the disjoint research efforts in relevant fields unified under the awareness of global constraints, objectives and application scenarios ?} 

\noindent\textbf{ii) How can we advocate for mathematical models rich in expressivity thus enabling the exploration, identification, and formulation of IoT system optimization challenges from a global perspective that goes well beyond the reach of fragmented research efforts in specific fields ? }

\noindent\textbf{iii) How do such models lift us to reach the possible solutions to these challenges and make predictions of new IoT paradigms ?} \\
\indent To discuss these questions, we position this paper not as a simply summarized list of well-known problems in IoT community but as a brief investigation of unexplored challenges based on our proposed mathematical model of the IoT system, hence a \textit{model-centric investigation}. It should be noted that even though these identified challenges exist objectively and do not change as a function of how we describe them, yet the challenges as well as their nature might not be well understood or even recognized without the help of a proper mathematical characterization (i.e., modeling and formulation based on it) that encodes a sufficient set of elements in related domains (i.e., rich expressivity). The formulation of these challenges are tightly connected with our understanding of the problem space, which further directly translates to how we could reason about the feasibility domain of the problem, the possible design of algorithms for solving it and the quality of the solutions. All these are added to the importance of the aforementioned challenges, thus motivating our discussion throughout this work.

% PhD proposal 
Recent astonishing advances on sensing modalities, data mining, communication and computation design methodologies enable us to encode multi-system interactions in both physical and cyber domains by the construction of cyber physical systems (CPS). CPS principles play the primary role in building up the foundations for a variety of emerging domains such as precise and personalized medicine, intelligent and autonomous vehicles and internet-of-things (IoT). However, enabling a closed-loop control over the CPS systems requires a systematic understanding of the synergistic coupling between physical and cyber processes. This poses major challenges in construction of the theoretical and computational foundations for making CPS a reachable reality.

In the computational domain, advanced scientific and engineering investigations in CPS systems generate exorbitantly large amount of data that is much beyond our accessible computing power. For instance, the data generated by genomics applications in precise medicine are projected to exceed any other Big Data applications with scale of zeta-byte by year of 2025. Processing astronomical volume of data requires computing capabilities ranging from 2-1000 trillion CPU hours due to the excessive resource locking, data contention and poorly exploited fine-grain parallelism in well-established computing paradigms (e.g., CPU, GPGPU). The fact that CPS applications are rich in non-stationary behaviors with lack of regular communication structures (e.g., physiological and biological system learning and control) exacerbates the drawbacks of parallelization techniques, memory wall, or the need for multiple instruction multi-data (MIMD) execution. Overcoming these challenges requires revolutionary design methodologies that break the hardware-software boundary to provide algorithmic breakthroughs for data movement, energy efficiency, memory wall, and extreme concurrency challenges. 

In the theoretical domain, the mathematical understanding, description, prediction and control of physical processes poses critical challenges in design and optimization of CPS systems that are highly reliable and robust with bounded performance margins. The irregular and non-stationary dynamics of CPS applications contradicts the traditional statistical memoryless assumption (e.g., Markovian processes, short-term memory) and cannot be captured by integer-order calculus. Instead, they exhibit long-term memory and can only be accurately modeled by fractal-order differential equations. For instance, inferring causal and higher-order complex relations in physiological and biological systems (e.g., cardiovascular, muscular or neural systems) from unstructured time varying data requires a spatio-temporal fractal model that considers multivariate dependency and a power-law temporal correlations. The failure to capture these properties of CPS applications in a robust and real-time fashion not only prevents its practical employment, but can also lead to irreversible consequences especially in presence of noisy and limited measurements, system anomalies, environmental uncertainties and malicious attacks. 

To tackle these challenges, my PhD research focuses on the following primary research tasks:
(1). How can we leverage a new design methodology to be capable of processing and mining the exscale data with optimal computation and communication architectures for CPS applications of tremendous heterogeneity? 
(2). What are the compact yet accurate construction of mathematical models that are able to capture the non-Markovian, non-stationary and non-linear system dynamics? 
To address the first task, we propose radical design methodologies for the exascale computation and communication architectures, making the following contributions: (1) We develop a LLVM-compiler framework based architecture-independent application profiling engine to learn the CPS application behaviors and inter-dependent structure in real-time. (2) We propose a novel model of computation (MoC) that describes the identified CPS applications behaviors and structures as a directed dynamical weighted graph (DDWG) in terms of required computation capabilities, data movement, storage requirements and timing characteristics to support the discovery of the optimal fine-grained parallelism and enables an automatic top-down synthesis of best-fit architectures with 10x performance improvement given the same energy budget. (2) To tackle the dynamic nature of CPS applications (e.g., adaptive multi-scaling, coupled multi-scale multi-physics), we develop traffic-aware routing algorithm with the support for node-cooperation network-coding communication and sub-modular algorithm with bounded optimality for NoC reconfiguration to achieve 100-fold throughput improvement and enable energy-efficient terabyte (TB) data movement. (3) We propose a general hardware accelerator synthesis framework by combining (1) and (2) to facilitate automated application-to-architecture top-down design methodology for CPS systems.

 To address the second task, we embrace the novel mathematical framework in modeling and control theory. More specifically: (1) We pioneer a dynamical systems approach to the problem of tracking both muscular and neural activity, and demonstrated that the current modeling approaches (e.g., ARMA. ARFIMA) of muscular and neural activities either ignored the long-term dependency in system dynamics or spatial coupling among system components. (2) We propose to describe the evolution of biological systems by construction of discrete-time fractional-order systems and demonstrate much improved performance in terms of goodness-of-fit against modeling approaches with short-term memory assumptions. (3) We show that the problem of determining the minimum number of sensors such that the global dynamics described by the fractional-order differential equation can be recovered from the collected data (i.e., observability) is NP-hard. Subsequently, we propose a polynomial algorithm to obtain suboptimal solutions with optimality guarantees and are robust in presence of modeling errors, process and measurement noise. (4) We investigate the state-of-art wearables that monitors brain activities (e.g., Emotiv) and show that the current placement of EEG sensors on the scalp is not optimal with current setup.

